\documentclass[12pt, a4paper]{article}

% CONFIGURACIÓN BÁSICA
\usepackage[spanish]{babel}
\usepackage[utf8]{inputenc}
\usepackage[T1]{fontenc}
\usepackage{lmodern}
\usepackage{csquotes}

% FORMATO Y DISEÑO
\usepackage{geometry}
\geometry{a4paper, margin=2.5cm, headheight=14pt}
\usepackage{fancyhdr}
\usepackage{setspace}
\onehalfspacing
\usepackage{enumitem}
\usepackage{graphicx}
\usepackage{float}
\usepackage{caption}

% MATEMÁTICAS Y SÍMBOLOS
\usepackage{amsmath}
\usepackage{amsfonts}
\usepackage{amssymb}

% CÓDIGO Y LISTADOS
\usepackage{listings}
\usepackage{xcolor}

% CONFIGURACION YAML

\lstdefinelanguage{yaml}{
    keywords={},
    sensitive=false,
    comment=[l]{\#},
    morestring=[b]",
}


% ENLACES Y REFERENCIAS
\usepackage{hyperref}
\hypersetup{
    colorlinks=true,
    linkcolor=blue,
    filecolor=magenta,      
    urlcolor=cyan,
    pdftitle={Configuración de red Raspberry y VM},
}

% CONFIGURACIÓN DE LISTINGS
\lstset{
    basicstyle=\ttfamily\small,
    keywordstyle=\color{blue},
    stringstyle=\color{red},
    commentstyle=\color{green},
    showstringspaces=false,
    numbers=left,
    numberstyle=\tiny\color{gray},
    frame=single,
    breaklines=true,
    morekeywords={systemctl, journalctl, service, ip, sudo, qemu-system-x86_64, k3s, kubectl},
    inputencoding=utf8,
    extendedchars=true,
    literate={á}{{\'a}}1 {é}{{\'e}}1 {í}{{\'i}}1 {ó}{{\'o}}1 {ú}{{\'u}}1 {ñ}{{\~n}}1
}

% METADATOS DEL DOCUMENTO
\title{\textbf{Configuración de red entre Raspberry Pi y VM} \\
\large Administración de Sistemas Operativos - RA 2 - CE g \\
Unidad Didáctica 1: Redes y virtualización}
\author{Alvaro Vazquez Vazquez \\ I.E.S. Fernando Aguilar Quignon}
\date{\today}

% CONFIGURACIÓN DE PIE DE PÁGINA
\pagestyle{fancy}
\fancyhf{}
\fancyhead[L]{\footnotesize Unidad Didáctica 1: Redes y virtualización}
\fancyhead[C]{\footnotesize ASO - RA 2 - CE g}
\fancyhead[R]{\footnotesize \thepage}
\fancyfoot[L]{\footnotesize Alvaro Vazquez Vazquez}
\fancyfoot[C]{\footnotesize I.E.S. Fernando Aguilar Quignon}
\fancyfoot[R]{\footnotesize \today}
\renewcommand{\headrulewidth}{0.4pt}
\renewcommand{\footrulewidth}{0.4pt}
\setlength{\headsep}{15pt}

\begin{document}

% PORTADA
\begin{titlepage}
    \centering
    \vspace*{2cm}
    {\Huge \textbf{Configuración de red entre Raspberry Pi y VM} \par}
    \vspace{0.5cm}
    {\Large \textbf{Administración de Sistemas Operativos} \par}
    \vspace{0.5cm}
    {\large 1ª Evaluación - RA 2 - CE g \par}
    \vspace{1cm}
    {\large \textbf{Unidad Didáctica 1:} \\ Redes y virtualización \par}
    \vspace{2cm}
    {\Large Alvaro Vazquez Vazquez \par}
    \vspace{0.5cm}
    {\large \today \par}
    \vspace{2cm}
    \vfill
    {\large I.E.S. Fernando Aguilar Quignon \par}
    {\small C/Conil de la Frontera, 3 \par}
    {\small CP 11010, Cádiz \par}
\end{titlepage}

\tableofcontents
\clearpage

\section{Introducción}
Este documento describe la configuración de red entre una máquina host, una VM QEMU y una Raspberry Pi para implementar un clúster con K3s. Se incluyen los comandos necesarios para crear un \textbf{bridge} en la máquina host y habilitar el acceso a Internet desde la VM, así como la preparación para que la Raspberry actúe como nodo \textbf{slave}.

\section{Configuración de red}

\subsection{Creación de un puente (bridge) en la máquina host}

\textbf{Pasos para crear br0 y asociar la interfaz física enp5s0:}

\begin{lstlisting}[language=bash, caption=Creación del bridge br0]
sudo ip link add name br0 type bridge
sudo ip link set enp5s0 master br0
sudo ip link set br0 up
sudo ip addr add 192.168.1.33/24 dev br0
sudo ip route add default via 192.168.1.1
sudo ip addr flush dev enp5s0
\end{lstlisting}

\textbf{Verificación de interfaces:}

\begin{lstlisting}[language=bash, caption=Verificación de interfaces]
ip a show enp5s0
ip a show br0
ping -c 3 8.8.8.8
\end{lstlisting}

\textbf{Notas importantes:}
\begin{itemize}
    \item La interfaz \texttt{enp5s0} queda asociada al bridge \texttt{br0}.
    \item \texttt{br0} toma la IP de la máquina host para proveer conectividad a la VM.
    \item Se comprueba la conectividad a Internet mediante ping a un DNS público.
\end{itemize}

\subsection{Configuración de red en la VM (Netplan)}

En la VM de QEMU se configura la red de forma estática mediante Netplan. El archivo \texttt{/etc/netplan/00-installer-config.yaml} tiene la siguiente configuración:

\begin{lstlisting}[language=yaml, caption=Archivo de Netplan de la VM QEMU]
network:
  version: 2
  renderer: networkd
  ethernets:
    ens3:
      dhcp4: no
      addresses: [192.168.1.101/24]
      gateway4: 192.168.1.33
      nameservers:
        addresses: [8.8.8.8, 8.8.4.4]
\end{lstlisting}

\textbf{Captura de pantalla:}  

\begin{figure}[H]
    \centering
    \includegraphics[draft]{1.png}
    \caption{Configuración de red de la VM QEMU (Netplan)}
    \label{fig:netplan_vm}
\end{figure}


\subsection{Configuración de QEMU para usar el bridge}

Archivo de script \texttt{qemuUS.sh}:

\begin{lstlisting}[language=bash, caption=Script para lanzar VM con QEMU usando br0]
sudo qemu-system-x86_64 \
  -enable-kvm \
  -m 4096 \
  -smp 2 \
  -cpu host \
  -hda /home/archi/Downloads/ubuntu-server.qcow2 \
  -boot d \
  -netdev bridge,id=net0,br=br0 \
  -device virtio-net-pci,netdev=net0 \
  -vga virtio
\end{lstlisting}

\textbf{Explicación:}
\begin{itemize}
    \item \texttt{-netdev bridge,id=net0,br=br0} conecta la VM al bridge creado.
    \item \texttt{-device virtio-net-pci,netdev=net0} proporciona la interfaz virtual en la VM.
\end{itemize}

\subsection{Configuración persistente con systemd-networkd}

\textbf{Archivo \texttt{/etc/systemd/network/br0.netdev}:}
\begin{lstlisting}[language=bash]
[NetDev]
Name=br0
Kind=bridge
\end{lstlisting}

\textbf{Archivo \texttt{/etc/systemd/network/enp5s0.network}:}
\begin{lstlisting}[language=bash]
[Match]
Name=enp5s0

[Network]
Bridge=br0
\end{lstlisting}

\textbf{Archivo \texttt{/etc/systemd/network/br0.network}:}
\begin{lstlisting}[language=bash]
[Match]
Name=br0

[Network]
Address=192.168.1.33/24
Gateway=192.168.1.1
DNS=8.8.8.8
\end{lstlisting}

\textbf{Habilitar servicios:}
\begin{lstlisting}[language=bash]
sudo systemctl enable --now systemd-networkd
sudo systemctl enable --now systemd-resolved
\end{lstlisting}

\section{Configuración de K3s}

\subsection{Instalación y configuración del nodo Master (VM)}

\textbf{Instalación de K3s en la VM Master:}

\begin{lstlisting}[language=bash]
curl -sfL https://get.k3s.io | sh -
sudo k3s kubectl get nodes
\end{lstlisting}

\textbf{Obtención del token para unir nodos:}

\begin{lstlisting}[language=bash]
sudo cat /var/lib/rancher/k3s/server/node-token
\end{lstlisting}

\subsection{Unión de la Raspberry Pi como nodo Agent (Slave)}

\textbf{Comando para unir la Raspberry al Master:}

\begin{lstlisting}[language=bash]
curl -sfL https://get.k3s.io | K3S_URL=https://192.168.1.101:6443 K3S_TOKEN=<TOKEN_DEL_MASTER> sh -
\end{lstlisting}

\textbf{Verificación de nodos en el Master:}

\begin{lstlisting}[language=bash]
sudo k3s kubectl get nodes
# Deberías ver algo así:
# NAME   STATUS   ROLES                  AGE   VERSION
# k3s    Ready    control-plane,master   42h   v1.33.5+k3s1
# raspberry Ready <none>                 5s    v1.33.5+k3s1
\end{lstlisting}

\section{Conclusión}

La creación del bridge permite que la VM tenga acceso directo a la red local y a Internet, facilitando la instalación de K3s y la integración con la Raspberry Pi como nodo slave. Esta configuración garantiza conectividad estable y control centralizado mediante la VM.

\clearpage
\section{Bibliografía}
\begin{thebibliography}{9}
\bibitem{qemu_bridge}
QEMU Documentation, ``Networking with QEMU'', 2023. \url{https://www.qemu.org/docs/master/network/}
\bibitem{k3s}
Rancher Labs, ``K3s Lightweight Kubernetes'', 2023. \url{https://k3s.io/}
\bibitem{systemd_networkd}
Systemd Documentation, ``systemd-networkd'', 2023. \url{https://www.freedesktop.org/software/systemd/man/systemd-networkd.service.html}
\end{thebibliography}

\end{document}
